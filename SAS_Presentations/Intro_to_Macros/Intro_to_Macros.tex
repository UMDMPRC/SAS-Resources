\documentclass{beamer}
\usepackage[utf8]{inputenc}
\usepackage{verbatim}
\usepackage{tikz}
\usepackage{hyperref}
\usepackage{graphicx}
\usepackage{hyperref}
\setbeamertemplate{footline}[frame number]
\title{Intro to Macros}
\author{\texorpdfstring{Heide Jackson \newline\url{heidej@umd.edu}}{Author}}
\institute{University of Maryland Population Research Center}

\date{August 2019}

\begin{document}
\maketitle
\begin{frame}{High Level Things to Know}
\begin{itemize}
\item Macros are objects or functions called within SAS to streamline coding and reduce redundancies.
\item This presentation will introduce macro variables and functions and how they can be useful for data manipulation.
\end{itemize}
\end{frame}


\begin{frame}[fragile]{Macro Variables}

\begin{itemize}
\item Macro variables are a way of representing an object that can later be called into a data set.
\item Macro variables usually have a \& preceeding their name.
\item Macro variables are often concluded with a .
\item Let is a common way of defining the macro.
\item Put displays the content of a macro variable.
\item Let's see this in action:
\end{itemize}
\begin{verbatim}
%let doors=HELLO;
%put &doors.;
\end{verbatim}
\item Macro variables can hold multiple words numbers or other types of objects that we may want to store
\end{frame}

\begin{frame}[fragile]{When Might Macro Variables be Useful?}
\begin{itemize}
    \item When working with data that is repetitive and just one element might change every year (for example 2018data, 2019data etc.)--we could create a macro shorthand for referring to the year.
    \item When working with big data sets or lists of variables. Here's an example:
\end{itemize}
\begin{verbatim}
    %let keep= my long list of various things;
    %let drop= things i do not need anymore;
    data savedata (keep=&keep.);
    set loaddata (drop=&drop.);
    run;
    
\end{verbatim}
\end{frame}
\begin{frame}[fragile]{Macro Variable Types}
\begin{itemize}
    \item Macros can be defined globally (available throughout the sas session) or locally, available only in the context of a macro function.
    \item Additionally, sas has some system defined macro variables and users can define their own macro variables.
    \item To see all macros available in a session use \%put \_ALL\_;
\end{itemize}
\end{frame}
\begin{frame}[fragile]{Macro Functions}
\begin{itemize}
    \item Local macro variables are often called in the context of macro functions.
    \item Functions are a way of calling code that may get used on multiple occasions.
    \item Let's say we want to recode all dummy variables to be 0 or 1.  Here's a macro to do it:
    \begin{verbatim}
%macro dummy(list);
%let count=%sysfunc(countw(&list.));
%do i=1 %to &count;
%let var=%scan(&list,&i);
if &var=2 then &var=0;
 %end;
%mend dummy;
    \end{verbatim}
\end{itemize}
\end{frame}

\begin{frame}[fragile]{Macro Functions}{Unpacking the Function}
\begin{itemize}
    \item \%macro begins the macro function
    \item "dummy" is the name of the function for when we call it later.
    \item (item) contains the local macro variables used within the function.  Note: when the function is used they can have different names from whatever is specified when the macro is created.
    \item \%sysfunc, and \%scan are internal sas functions for looking within macro objects.
    \item \%do \%to and \%end indicate we are looping in a macro environment.
    \item mend macro name indicates the macro is completed.
\end{itemize}
\end{frame}
\begin{frame}[fragile]{Calling the Function}
\begin{itemize}
\item Once defined, this function can be used by using \% function name.
\begin{verbatim}
data cars; set sashelp.cars;
Asia=1; if origin="Asia" then Asia=2;
Europe=1; if origin="Europe" then Europe=2;
USA=1; if origin="USA" then USA=2;
run;
%let mylist=Asia Europe USA;
data cars2; set cars;
%dummy(&mylist.);
run;
\end{verbatim}
\end{itemize}
\end{frame}
\begin{frame}[fragile]{Debugging the Function}
\begin{itemize}
    \item SAS has two options, mlogic and mprint that can help unpack what a function does and how it interprets macro objects.
    \item These options can sometimes be quite verbose, so they should be mainly used for debugging purposes.
\end{itemize}
\end{frame}
\begin{frame}[fragile]{Defining Macro Variables in SQL}
\begin{itemize}
    \item SQL can be very useful for defining macro variables and macro lists.
    \item The into command of SQL can help us do this.
    \begin{verbatim}
        proc sql; 
select distinct origin into :orig separated by " "
from sashelp.cars;
quit;
    \end{verbatim}
\end{itemize}
\end{frame}

\begin{frame}[fragile]{Using Macros to Read in Other SAS Scripts}
\begin{itemize}
\item The include command can be used to reference libnames or macros stored in other files.
\item This can often be useful if there are a list of functions or libraries that you want to call without adding bulk and redundancy to your current sas script.
\begin{verbatim}
/*Reads in library locations stored in a 
sas file and makes them available in session*/
%include "somelocations\libnames.sas";
\end{verbatim}
\end{itemize}
\end{frame}
\end{document}
