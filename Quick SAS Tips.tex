\documentclass{beamer}
\usepackage[utf8]{inputenc}
\usepackage{verbatim}
\usepackage{tikz}
\usepackage{hyperref}
\setbeamertemplate{footline}[frame number]
\title{Quick SAS Tips}
\author{\texorpdfstring{Heide Jackson \newline\url{heidej@umd.edu}}{Author}}
\institute{University of Maryland Population Research Center}

\date{August 2019}

\begin{document}
\maketitle
\begin{frame}{High Level Things to Know}
\begin{itemize}
\item SAS is a statistical software package well suited to handling large data sets
\item Federal data sets are commonly stored as SAS data files (these end in the extension sas7bdat)
\item SAS can handle multiple data sets simultaneously and keep them in its remote memory.
\item SAS is not case sensitive and has extensions for using other languages particularly SQL and Python.
\end{itemize}
\end{frame}


\begin{frame}{SAS Versions}

\item SAS 9,4 is available for download for University of Maryland affiliates.
\item SAS can be accessed online via SAS Studio

\end{frame}

\begin{frame}[fragile]{The basics of doing anything in SAS}{Loading in Data}
\begin{itemize}
\item Most data is loaded, modified and saved within a data statement.  This corresponds to: 
\begin{verbatim}
data new; /*Saves new data*/
set existing; /*Loads existing data set*/
run; /*Runs or compiles the statement*/
\end{verbatim}
\item Performing analysis on an existing data follows the following structure:
\end{itemize}
\end{frame}


\begin{frame}[fragile]{The basics of doing anything in SAS}{Analyzing Data}
\begin{itemize}
\item Once the data is loaded, analysis can be run using the following general structure.
\begin{verbatim}
proc somecommand data=existing;
intermediate statements;
run;
\end{verbatim}
\end{itemize}
\end{frame}



\begin{frame}[fragile]{Finding and Working with Data}
\begin{itemize}
\item By default SAS looks for data and stores data within a temporary working directory.  This can be found in SAS's explorer window.
\item Data sets also exist in a directory internal to sas called sashelp.
\item Users can tell SAS to look at external directories with a libname statement.
\begin{verbatim}
libname somename "/specified/full/location";
\end{verbatim}
\end{itemize}
\end{frame}

\begin{frame}[fragile]{Working with Outside Data Locations}
\begin{itemize}
\item Once a libname is specified, data can be loaded in and saved in other locations.
\item Data can be saved by adding such a location to the data step.

\item Here's a hypothetical example:
\begin{verbatim}
libname save "/specified/full/location";
libname load "/specified/full/location2";

data save.data;
set load.read;
run;
\end{verbatim}
\end{itemize}
\end{frame}


\end{frame}
\begin{frame}[fragile]{Exporting Data}
\begin{itemize}
    \item Data can be easily exported to other programs and format.
    \item Two formats frequently used are xlsx (excel) and dta (stata).
    \begin{verbatim}
    proc export data=somesasdata filename="\full 
    location\
    dataname.extension" dbms=extension;
    /*Example of common extensions 
    are .xlsx or .dta*/
    run;
        \end{verbatim}
    \end{itemize}
    \end{frame}


\begin{frame}{General Suggestions for Getting Started}
\begin{itemize}

    \item The commands shown here can be entered via SAS's program editor.
        \item The program editor will be the script that allows users to create and reproduce data and results.
        \item Always use the log file to confirm that code was executed correctly.
        \item Unless data needs to be shared, there isn't a need to save it to a physical location.  The working directory works.
        \item It is good practice to use a different name for data in the data step versus the set step--this prevents data from being overwritten.

\end{itemize}
\end{frame}



\end{document}
